\documentclass[dvipdfmx]{beamer}
%テーマの設定
\usetheme{metropolis}

%パッケージの設定
\usepackage{graphicx}
\usepackage{ulem}
\usepackage{pxjahyper}
\usepackage{hyperref}
\usepackage{tikz}
\usetikzlibrary{arrows.meta,positioning,shapes}
\usepackage{listings, jlisting}
\usepackage{xcolor}

%ソースコード設定
\lstset{
  basicstyle={\ttfamily\footnotesize},            % 基礎の文字のフォント設定
  identifierstyle={\color{identifier}},           % 変数名などのフォント設定
  commentstyle   ={\color{comment}},              % コメントのフォント設定
  keywordstyle   =[1]{\color{keyword}},           % 予約語のフォント設定
  stringstyle    ={\ttfamily\color{string}},      % 文字列のフォント設定
  tabsize=2
}

%フォント設定
\renewcommand{\kanjifamilydefault}{\gtdefault}

%メタ設定
\title{IoT×アイディア プログラミング}
\subtitle{アイディアをカタチに}
\author[Jin]{Jin}
\date[\today]{\today}

\begin{document}

\begin{frame}[plain]
  \titlepage
\end{frame}

\begin{frame}{はじめに}
    \begin{large}目的\end{large}
    \begin{itemize}
        \setlength{\itemsep}{5mm}
        \item TurtlePico開発ボードを使用し、IoTに関する知識に加え、\\アイディアをカタチにできる技術を習得する。
        \item ソフトウェアとハードウェア双方の知識を習得し、\\幅広いレイヤーでの開発を行えるようになる。
        \item よりプログラミングを扱いやすいようにするため、\\開発環境はVSCodeによるMicropythonを使用する。
    \end{itemize}
\end{frame}

\begin{frame}{カリキュラム}
    \begin{tikzpicture}
      [every node/.style={rounded corners,inner sep=5pt, minimum width=100pt,minimum height=35pt},
      pre-subj/.style={fill=blue!20},post-subj/.style={fill=black!20},overlay, remember picture]
      \node[pre-subj] (a) at (0.13\textwidth,0.1\textheight){環境構築};
      \node[pre-subj,right=15pt of a] (b) {Pythonの基礎};
      \node[pre-subj,right=15pt of b] (c) {マイコンの基礎};
      \node[pre-subj,below=30pt of c] (d) {モーター基礎};
      \node[pre-subj,left=15pt of d] (e) {IoT基礎};
      \node[pre-subj,left=15pt of e] (f) {ものづくりへの応用};
      \foreach \x / \y in {a/b,b/c,c/d,d/e,e/f}{
        \draw[->,>={Stealth[round]},line width=.5mm] (\x) -- (\y);
      }
    \end{tikzpicture}
\end{frame}

\section{環境構築}

    \begin{frame}{開発環境とは}
      プログラミングにはツールがたくさん存在しており、それらを導入することを「環境構築」と呼ぶ。
      \begin{description}
        \item[\alert{テキストエディタ}] \mbox{}\\ プログラムを直接記述するツール。\\今回はMicrosoftのVSCodeを使用。
        \item[\alert{コンパイラ}] \mbox{}\\ プログラムを機械語に変換するツール。\\C言語などで使用される。
        \item[\alert{インタプリタ}] \mbox{}\\ プログラムを逐次実行するツール。\\今回はVSCode拡張機能「MicroPico」内蔵のPerlを使用。
      \end{description}
    \end{frame}
    \begin{frame}{Pythonのインストール}
      Pythonのバージョンは3.9以上のものをインストールする。ダウンロードリンクは以下の通り。
      \begin{itemize}
        \item Windows: \url{https://www.python.org/downloads/windows/}
        \item Mac: \url{https://www.python.org/downloads/mac-osx/}
        \item Linux: \url{https://www.python.org/downloads/source/}
      \end{itemize}
      \vfill
      ダウンロードができたらインストーラを実行し、指示に従って\alert{インストール}する。\\
      パスの追加が必要なため、この時にインストールフォルダを控えておくようにする。
    \end{frame}
    \begin{frame}{環境変数の追加}
      今回はWindowsの場合のみ説明する。
      \begin{enumerate}
        \item システム環境変数と検索し、環境変数設定画面を開く
        \item 「環境変数」をクリック
        \item システム環境変数のPathを選択し、編集をクリック
        \item 「編集」をクリックしてPythonのインストール先を追加
      \end{enumerate}
    \end{frame}
    \begin{frame}{VSCodeのインストール}
        ダウンロードリンクは \url{https://code.visualstudio.com/download}\\
        ダウンロードしたインストーラの指示に従ってインストールする。
    \end{frame}
    \begin{frame}{拡張機能の設定}
        
    \end{frame}

\section{Python基本のキ}
    \begin{frame}{変数}
        
    \end{frame}
    \begin{frame}{演算}
        
    \end{frame}
    \begin{frame}{if文}

    \end{frame}
    \begin{frame}{for文}
        
    \end{frame}
\end{document}