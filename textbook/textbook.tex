\documentclass[dvipdfmx]{beamer}
%テーマの設定
\usetheme{metropolis}

%パッケージの設定
\usepackage{graphicx}
\usepackage{ulem}
\usepackage{pxjahyper}
\usepackage{hyperref}
\usepackage{tikz}
\usetikzlibrary{arrows.meta,positioning,shapes}
\usepackage{listings, jlisting}
\usepackage{xcolor}

%ソースコード設定
\lstset{
  basicstyle={\ttfamily\footnotesize},            % 基礎の文字のフォント設定
  identifierstyle={\color{identifier}},           % 変数名などのフォント設定
  commentstyle   ={\color{comment}},              % コメントのフォント設定
  keywordstyle   =[1]{\color{keyword}},           % 予約語のフォント設定
  stringstyle    ={\ttfamily\color{string}},      % 文字列のフォント設定
  tabsize=2
}

%フォント設定
\renewcommand{\kanjifamilydefault}{\gtdefault}

%メタ設定
\title{IoT×アイディア プログラミング}
\subtitle{アイディアをカタチに}
\author[Jin]{Jin}
\date[\today]{\today}

\begin{document}

\begin{frame}[plain]
  \titlepage
\end{frame}

\begin{frame}{はじめに}
    \begin{large}目的\end{large}
    \begin{itemize}
        \setlength{\itemsep}{5mm}
        \item TurtlePico開発ボードを使用し、IoTに関する知識に加え、\\アイディアをカタチにできる技術を習得する。
        \item ソフトウェアとハードウェア双方の知識を習得し、\\幅広いレイヤーでの開発を行えるようになる。
        \item よりプログラミングを扱いやすいようにするため、\\開発環境はVSCodeによるMicropythonを使用する。
    \end{itemize}
\end{frame}

\begin{frame}{カリキュラム}
    \begin{tikzpicture}
      [every node/.style={rounded corners,inner sep=5pt, minimum width=100pt,minimum height=35pt},
      pre-subj/.style={fill=blue!20},post-subj/.style={fill=black!20},overlay, remember picture]
      \node[pre-subj] (a) at (0.13\textwidth,0.1\textheight){\hyperlink{devstruct}{環境構築}};
      \node[pre-subj,right=15pt of a] (b) {\hyperlink{python}{Pythonの基礎}};
      \node[pre-subj,right=15pt of b] (c) {\hyperlink{micon}{マイコンの基礎}};
      \node[pre-subj,below=30pt of c] (d) {\hyperlink{motor}{モーター基礎}};
      \node[pre-subj,left=15pt of d] (e) {\hyperlink{iot}{IoT基礎}};
      \node[pre-subj,left=15pt of e] (f) {\hyperlink{create}{ものづくりへの応用}};
      \foreach \x / \y in {a/b,b/c,c/d,d/e,e/f}{
        \draw[->,>={Stealth[round]},line width=.5mm] (\x) -- (\y);
      }
    \end{tikzpicture}
\end{frame}

\section*{環境構築}
    \begin{frame}[plain,noframenumbering]
        \label{devstruct}
        \sectionpage
    \end{frame}

    \begin{frame}{開発環境とは}
        
    \end{frame}
    \begin{frame}{Pythonのインストール}
        
    \end{frame}
    \begin{frame}{VSCodeのインストール}
        
    \end{frame}
    \begin{frame}{拡張機能の設定}
        
    \end{frame}

\section{Python基本のキ}
    \begin{frame}{変数}
        
    \end{frame}
    \begin{frame}{演算}
        
    \end{frame}
    \begin{frame}{if文}

    \end{frame}
    \begin{frame}{for文}
        
    \end{frame}
\end{document}